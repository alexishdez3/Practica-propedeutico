\documentclass{article}
\usepackage[english]{babel}
\usepackage[utf8]{inputenc}
\usepackage[T1]{fontenc}
\usepackage{enumerate}

\usepackage[
  top=3cm,
  bottom=3cm,
  left=2cm,
  right=2cm,
  heightrounded,
]{geometry}

\author{Luis Martínez}
\title{Enumeración}
\date{\today}

\begin{document}

\maketitle

Plantilla de \LaTeX.

% https://www.lipsum.com
\section{Ejemplos de enumeración en Latex}
\begin{enumerate}
\item Este es el primer párrafo en el primer nivel
\item Este es el segundo reglón
  \begin{enumerate}
  \item Queremos ver qué pasa al anidar en el segundo nivel
  \item Este es el segundo reglón
  \end{enumerate}
\end{enumerate}

\begin{enumerate}[(a)]
\item Este es otro ejemplo de enumeración
\item aquí hay otro renglón
\end{enumerate}

\begin{enumerate}[I.]
\item Ejemplo de incisos con números romanos
	\item Segundo inciso con números romanos
\end{enumerate}

Este es el ejemplo de una formula matemática
$$ X_i + Y_i = 6 $$

Ejemplo de una fracción 
$$\frac{x_6}{y_8}$$\\


Ejemplo de signo especial $\alpha$ 

Ejemplo con raíz cuadrada $$\frac{-\beta \pm \sqrt[2]{\beta^2-4\alpha\gamma}}{2\alpha}$$




\end{document}

